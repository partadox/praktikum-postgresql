\chapter{Implementasi Partitioning pada PostgreSQL}
\setcounter{section}{0}
\section{Pendahuluan}
Partitioning adalah teknik untuk membagi tabel besar menjadi beberapa bagian yang lebih kecil secara fisik, namun tetap terlihat sebagai satu tabel secara logis. Dalam sistem database dengan volume data yang besar, partitioning menjadi strategi penting untuk meningkatkan performa, maintainability, dan skalabilitas.

Ketika volume data tumbuh, performa query dapat menurun signifikan karena database harus memproses lebih banyak data. Dengan partitioning, database dapat membatasi jumlah data yang diakses saat menjalankan query, sehingga meningkatkan efisiensi operasi.

\subsection{Tujuan Praktikum}
Dilaksanakannya praktikum partitioning, praktikan diharapkan mampu:
\begin{enumerate}
    \item Memahami konsep partitioning pada database
    \item Mengimplementasikan berbagai tipe partitioning pada PostgreSQL
    \item Menganalisis dampak partitioning terhadap performa query
    \item Mempelajari strategi partitioning yang tepat untuk berbagai jenis data
\end{enumerate}

\section{Konsep Dasar Partitioning}

\subsection{Apa itu Partitioning?}
Partitioning adalah teknik untuk membagi tabel besar menjadi beberapa bagian yang lebih kecil secara fisik tetapi tetap terlihat sebagai satu tabel secara logis. Hal ini memungkinkan database untuk mengoptimalkan query dengan hanya mengakses partisi yang relevan dengan query, bukan seluruh tabel.

\subsection{Manfaat Partitioning}
\begin{enumerate}
    \item \textbf{Peningkatan Performa Query}: Query yang hanya mengakses subset data (misalnya satu bulan dari data time-series) dapat berjalan lebih cepat karena hanya mengakses partisi yang relevan
    \item \textbf{Efisiensi Maintenance}: Operasi maintenance seperti backup, restore, dan archiving dapat dilakukan pada level partisi
    \item \textbf{Optimalisasi Penyimpanan}: Data dengan pola akses berbeda dapat disimpan pada media penyimpanan yang berbeda (misalnya data terbaru pada SSD)
    \item \textbf{Paralelisme}: Query terhadap beberapa partisi dapat dieksekusi secara paralel
    \item \textbf{Penghapusan Data Efisien}: Menghapus data dalam jumlah besar dapat dilakukan dengan DROP PARTITION daripada DELETE yang lebih lambat
\end{enumerate}

\subsection{Kapan Menggunakan Partitioning}
Partitioning sangat berguna pada:
\begin{enumerate}
    \item Tabel dengan volume data sangat besar (puluhan GB atau lebih)
    \item Data dengan pola akses yang jelas berdasarkan kolom tertentu
    \item Data time-series yang memiliki pola pemasukan dan penghapusan teratur
    \item Tabel fakta pada data warehouse
    \item Sistem yang memerlukan high-availability untuk data terbaru sementara data historis dapat diakses lebih lambat
\end{enumerate}

\subsection{Kapan Tidak Menggunakan Partitioning}
Partitioning tidak selalu bermanfaat pada:
\begin{enumerate}
    \item Tabel dengan ukuran kecil atau menengah yang dapat diproses efisien tanpa partitioning
    \item Tabel yang sering di-join dengan tabel lain (trade-off performa join vs. partitioning)
    \item Sistem dengan pola query yang sulit diprediksi atau tidak selaras dengan strategi partitioning
    \item Ketika overhead administrasi partisi lebih besar dari manfaatnya
\end{enumerate}

\section{Jenis-jenis Partitioning di PostgreSQL}

PostgreSQL menyediakan beberapa tipe partitioning yang dapat dipilih sesuai dengan karakteristik data dan pola akses:

\subsection{Range Partitioning}
\begin{enumerate}
    \item Membagi data berdasarkan rentang nilai pada kolom tertentu
    \item Ideal untuk data time-series (tanggal, timestamp), nilai numerik berurutan
    \item Sangat efektif untuk query yang mengakses data dalam rentang tertentu
    \item Contoh: partisi data penjualan per bulan, per kuartal, atau per tahun
\end{enumerate}

\subsection{List Partitioning}
\begin{enumerate}
    \item Membagi data berdasarkan daftar nilai diskrit pada kolom tertentu
    \item Cocok untuk kolom dengan nilai terbatas dan terdefinisi
    \item Efektif untuk query yang mem-filter berdasarkan nilai spesifik
    \item Contoh: partisi berdasarkan region, status pesanan, kategori produk
\end{enumerate}

\subsection{Hash Partitioning}
\begin{enumerate}
    \item Membagi data berdasarkan hasil fungsi hash dari nilai kolom
    \item Mendistribusikan data secara relatif merata di semua partisi
    \item Cocok saat tidak ada pola akses yang jelas atau untuk distribusi beban
    \item Kurang optimal untuk query range, tetapi baik untuk paralelisme
    \item Contoh: partisi berdasarkan hash dari customer\_id untuk distribusi merata
\end{enumerate}

\begin{figure}[h]
	\centering
	\begin{tabular}{|l|p{10cm}|}
		\hline
		\textbf{Tipe Partitioning} & \textbf{Karakteristik dan Penggunaan} \\
		\hline
		Range & Membagi berdasarkan rentang nilai (tanggal, timestamp, numerik). Ideal untuk data time-series dan query range. Misalnya: data per bulan, per semester, atau per tahun. \\
		\hline
		List & Membagi berdasarkan daftar nilai diskrit. Cocok untuk kolom dengan nilai terbatas seperti region, status, kategori. Misalnya: data per provinsi, per status pesanan. \\
		\hline
		Hash & Membagi berdasarkan hasil fungsi hash. Distribusi data merata, cocok untuk paralelisme dan distribusi beban. Misalnya: hash dari customer\_id untuk distribusi merata. \\
		\hline
	\end{tabular}
	\caption{Jenis-jenis Partitioning di PostgreSQL dan Penggunaannya}
\end{figure}

\subsection{Sub-partitioning (Multi-level Partitioning)}
PostgreSQL juga mendukung partitioning bertingkat (sub-partitioning), di mana partisi dapat dibagi lagi menjadi sub-partisi. Hal ini memungkinkan fleksibilitas lebih tinggi dalam manajemen data:

\begin{enumerate}
    \item Partisi level pertama bisa menggunakan Range (misalnya, per tahun)
    \item Sub-partisi bisa menggunakan List (misalnya, per region atau status)
    \item Atau kombinasi lain dari jenis partitioning
\end{enumerate}

\section{Best Practices Implementasi Partitioning}

\begin{enumerate}
    \item \textbf{Pilih kolom partitioning yang sesuai} dengan pola akses query yang paling umum
    \item \textbf{Batasi jumlah partisi} menjadi jumlah yang dapat dikelola (tidak terlalu banyak)
    \item \textbf{Pertimbangkan partitioning default} untuk menangani data yang tidak masuk kategori partisi yang ada
    \item \textbf{Buat index pada setiap partisi} untuk kolom yang sering digunakan dalam query
    \item \textbf{Gunakan constraint exclusion} untuk memastikan query optimizer melewatkan partisi yang tidak relevan
    \item \textbf{Pertimbangkan tools otomatisasi} untuk mengelola partisi yang dibuat secara berkala
    \item \textbf{Uji performa} untuk memverifikasi bahwa partitioning memberikan manfaat yang diharapkan
    \item \textbf{Kombinasikan dengan strategi penyimpanan} (tablespace) untuk pengoptimalan lebih lanjut
\end{enumerate}

\section{Tahapan Praktikum}

\subsection{Persiapan}
\begin{enumerate}
	\item Pastikan container Docker PostgreSQL sudah berjalan
	\item Gunakan pgAdmin atau PSQL untuk mengakses database
	\item Pastikan data sudah diimpor menggunakan script yang disediakan
\end{enumerate}

\subsection{Implementasi Range Partitioning}
Pada bagian ini, kita akan membuat partisi tabel orders berdasarkan tanggal pesanan (order date):

\begin{lstlisting}[language=SQL]
	-- Buat tabel induk dengan deklarasi partisi
	CREATE TABLE orders_partitioned (
		order_id SERIAL,
		customer_id INTEGER NOT NULL,
		order_date TIMESTAMP NOT NULL,
		status VARCHAR(50) DEFAULT 'pending',
		total_amount DECIMAL(12, 2),
		shipping_address TEXT,
		payment_method VARCHAR(50),
		PRIMARY KEY (order_id, order_date),
		CONSTRAINT valid_status CHECK (status IN ('pending', 'processing', 'shipped', 'delivered', 'cancelled'))
	) PARTITION BY RANGE (order_date);
	
	-- Buat partisi untuk setiap kuartal tahun 2023
	CREATE TABLE orders_q1_2023 PARTITION OF orders_partitioned
		FOR VALUES FROM ('2023-01-01') TO ('2023-04-01');
	
	CREATE TABLE orders_q2_2023 PARTITION OF orders_partitioned
		FOR VALUES FROM ('2023-04-01') TO ('2023-07-01');
	
	CREATE TABLE orders_q3_2023 PARTITION OF orders_partitioned
		FOR VALUES FROM ('2023-07-01') TO ('2023-10-01');
	
	CREATE TABLE orders_q4_2023 PARTITION OF orders_partitioned
		FOR VALUES FROM ('2023-10-01') TO ('2024-01-01');
	
	-- Import data dari tabel orders asli
	INSERT INTO orders_partitioned
	SELECT * FROM orders;
\end{lstlisting}


\subsection{Implementasi List Partitioning}
Selanjutnya, kita akan mencoba partitioning berdasarkan status pesanan:

\begin{lstlisting}[language=SQL]
	-- Buat tabel induk dengan deklarasi partisi
	CREATE TABLE orders_by_status (
		order_id SERIAL,
		customer_id INTEGER NOT NULL,
		order_date TIMESTAMP NOT NULL,
		status VARCHAR(50) NOT NULL,
		total_amount DECIMAL(12, 2),
		shipping_address TEXT,
		payment_method VARCHAR(50),
		PRIMARY KEY (order_id, status)
	) PARTITION BY LIST (status);
	
	-- Buat partisi untuk setiap status
	CREATE TABLE orders_pending PARTITION OF orders_by_status
		FOR VALUES IN ('pending');
	
	CREATE TABLE orders_processing PARTITION OF orders_by_status
		FOR VALUES IN ('processing');
	
	CREATE TABLE orders_shipped PARTITION OF orders_by_status
		FOR VALUES IN ('shipped');
	
	CREATE TABLE orders_delivered PARTITION OF orders_by_status
		FOR VALUES IN ('delivered');
	
	CREATE TABLE orders_cancelled PARTITION OF orders_by_status
		FOR VALUES IN ('cancelled');
	
	-- Import data dari tabel orders asli
	INSERT INTO orders_by_status
	SELECT * FROM orders;
\end{lstlisting}

\subsection{Implementasi Hash Partitioning}
Berikutnya, kita akan mencoba partitioning berdasarkan hash dari customer id:

\begin{lstlisting}[language=SQL]
	-- Buat tabel induk dengan deklarasi partisi
	CREATE TABLE orders_by_customer (
		order_id SERIAL,
		customer_id INTEGER NOT NULL,
		order_date TIMESTAMP NOT NULL,
		status VARCHAR(50) DEFAULT 'pending',
		total_amount DECIMAL(12, 2),
		shipping_address TEXT,
		payment_method VARCHAR(50),
		PRIMARY KEY (order_id, customer_id),
		CONSTRAINT valid_status CHECK (status IN ('pending', 'processing', 'shipped', 'delivered', 'cancelled'))
	) PARTITION BY HASH (customer_id);
	
	-- Buat 4 partisi
	CREATE TABLE orders_by_customer_0 PARTITION OF orders_by_customer
		FOR VALUES WITH (MODULUS 4, REMAINDER 0);
	
	CREATE TABLE orders_by_customer_1 PARTITION OF orders_by_customer
		FOR VALUES WITH (MODULUS 4, REMAINDER 1);
	
	CREATE TABLE orders_by_customer_2 PARTITION OF orders_by_customer
		FOR VALUES WITH (MODULUS 4, REMAINDER 2);
	
	CREATE TABLE orders_by_customer_3 PARTITION OF orders_by_customer
		FOR VALUES WITH (MODULUS 4, REMAINDER 3);
	
	-- Import data dari tabel orders asli
	INSERT INTO orders_by_customer
	SELECT * FROM orders;
\end{lstlisting}

\subsection{Menganalisis Performa Partitioning}
Pada bagian ini, kita akan membandingkan performa query antara tabel biasa dan tabel yang menggunakan partitioning:

\begin{lstlisting}[language=SQL]
	-- Query pada tabel non-partisi
	EXPLAIN ANALYZE
	SELECT *
	FROM orders
	WHERE order_date BETWEEN '2023-06-01' AND '2023-06-30';
	
	-- Query pada tabel dengan partisi range
	EXPLAIN ANALYZE
	SELECT *
	FROM orders_partitioned
	WHERE order_date BETWEEN '2023-06-01' AND '2023-06-30';
\end{lstlisting}

\subsection{Maintenance Partisi}
Selanjutnya, kita akan mempelajari bagaimana melakukan maintenance terhadap partisi:

\begin{lstlisting}[language=SQL]
	-- Menambahkan partisi baru untuk tahun 2024
	CREATE TABLE orders_q1_2024 PARTITION OF orders_partitioned
		FOR VALUES FROM ('2024-01-01') TO ('2024-04-01');
	
	-- Menghapus data lama dari partisi tertentu
	TRUNCATE TABLE orders_q1_2023;
	
	-- Mengarsipkan partisi lama
	ALTER TABLE orders_partitioned DETACH PARTITION orders_q1_2023;
	-- Sekarang orders_q1_2023 adalah tabel mandiri yang dapat diarchive
\end{lstlisting}

\subsection{Implementasi Sub-partitioning (Multi-level Partitioning)}
Kita juga akan mengimplementasikan partitioning bertingkat:

\begin{lstlisting}[language=SQL]
	-- Buat tabel induk dengan deklarasi partisi bertingkat
	CREATE TABLE orders_multi_level (
		order_id SERIAL,
		customer_id INTEGER NOT NULL,
		order_date TIMESTAMP NOT NULL,
		status VARCHAR(50) NOT NULL,
		total_amount DECIMAL(12, 2),
		shipping_address TEXT,
		payment_method VARCHAR(50),
		PRIMARY KEY (order_id, order_date, status)
	) PARTITION BY RANGE (order_date);
	
	-- Buat partisi kuartal, yang juga akan dipartisi lagi
	CREATE TABLE orders_q1_2023_multi PARTITION OF orders_multi_level
		FOR VALUES FROM ('2023-01-01') TO ('2023-04-01')
		PARTITION BY LIST (status);
	
	-- Subpartisi berdasarkan status untuk Q1 2023
	CREATE TABLE orders_q1_2023_pending PARTITION OF orders_q1_2023_multi
		FOR VALUES IN ('pending');
	
	CREATE TABLE orders_q1_2023_processing PARTITION OF orders_q1_2023_multi
		FOR VALUES IN ('processing');
	
	CREATE TABLE orders_q1_2023_shipped PARTITION OF orders_q1_2023_multi
		FOR VALUES IN ('shipped');
	
	CREATE TABLE orders_q1_2023_delivered PARTITION OF orders_q1_2023_multi
		FOR VALUES IN ('delivered');
	
	CREATE TABLE orders_q1_2023_cancelled PARTITION OF orders_q1_2023_multi
		FOR VALUES IN ('cancelled');
	
	-- Import data
	INSERT INTO orders_multi_level
	SELECT * FROM orders 
	WHERE order_date BETWEEN '2023-01-01' AND '2023-03-31';
\end{lstlisting}

\section{Tugas Praktikum}

\begin{enumerate}
    \item Implementasikan partitioning pada salah satu tabel besar (orders atau products)
    
    \item Bandingkan performa query antara tabel non-partisi dan tabel dengan partisi untuk:
    \begin{itemize}
        \item Query yang mengakses satu partisi
        \item Query yang mengakses beberapa partisi
        \item Query yang mengakses seluruh tabel
    \end{itemize}
    
    \item Implementasikan minimal dua tipe partitioning berbeda (Range, List, atau Hash)
    
    \item Buat script untuk maintenance partisi secara otomatis:
    \begin{itemize}
        \item Menambahkan partisi baru secara otomatis
        \item Mengarsipkan partisi lama
    \end{itemize}
    
    \item Lakukan benchmark konkuren dan bandingkan throughput pada tabel dengan dan tanpa partisi
\end{enumerate}

\section{Pertanyaan Diskusi}

\begin{enumerate}
    \item Kapan sebaiknya menggunakan partitioning dan kapan tidak perlu?
    
    \item Bagaimana strategi partitioning yang tepat untuk:
    \begin{itemize}
        \item Data time-series dengan volume tinggi
        \item Data yang sering diakses berdasarkan wilayah geografis
        \item Tabel fakta pada data warehouse
    \end{itemize}
    
    \item Apa perbedaan antara partitioning dan sharding? Kapan sebaiknya menggunakan masing-masing?
    
    \item Apa kelebihan dan kekurangan dari masing-masing tipe partitioning (Range, List, Hash)?
    
    \item Bagaimana partitioning mempengaruhi performa JOIN operations?
\end{enumerate}

\section{Referensi}

\begin{itemize}
    \item \href{https://www.postgresql.org/docs/current/ddl-partitioning.html}{PostgreSQL Documentation: Table Partitioning}
    \item \href{https://www.postgresql.org/docs/current/ddl-partitioning.html#DDL-PARTITIONING-DECLARATIVE}{PostgreSQL: Declarative Partitioning}
    \item \href{https://wiki.postgresql.org/wiki/Table_partitioning}{PostgreSQL Wiki: Table Partitioning}
    \item \href{https://www.postgresql.org/docs/current/ddl-partitioning.html#DDL-PARTITIONING-IMPLEMENTATION}{PostgreSQL: Partition Pruning}
    \item \href{https://www.postgresql.org/docs/current/ddl-partitioning.html#DDL-PARTITIONING-CONSTRAINT-EXCLUSION}{PostgreSQL: Constraint Exclusion}
\end{itemize}