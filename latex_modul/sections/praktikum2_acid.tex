\chapter{Implementasi Transaction dan ACID pada PostgreSQL}
\setcounter{section}{0}
\section{Pendahuluan}
Transaksi (transaction) merupakan salah satu konsep fundamental dalam sistem basis data relasional yang menjamin integritas dan reliabilitas data. Dalam PostgreSQL, implementasi transaksi yang tepat dan pemahaman ACID properties sangat penting untuk mengembangkan aplikasi yang handal, terutama dalam lingkungan dengan konkurensi tinggi.

Transaksi adalah unit kerja yang mengubah data dari satu state ke state lainnya. Semua perubahan dalam transaksi harus berhasil secara lengkap, atau tidak ada yang dilakukan sama sekali (all-or-nothing). Konsep ini menjamin bahwa database tetap dalam keadaan konsisten meskipun terjadi kegagalan sistem.

\subsection{Tujuan Praktikum}
Dilaksanakannya praktikum transaction dan ACID, praktikan diharapkan mampu:
\begin{enumerate}
    \item Memahami konsep ACID dalam transaksi database
    \item Mengimplementasikan transaksi pada PostgreSQL
    \item Mempelajari tingkat isolasi transaksi
    \item Menganalisis fenomena konkurensi seperti dirty read, non-repeatable read, dan phantom read
    \item Mengimplementasikan strategi penanganan konkurensi tinggi
\end{enumerate}

\section{Konsep ACID dalam Transaksi Database}

ACID adalah akronim yang merepresentasikan empat properti penting dari transaksi database yang menjamin reliabilitas data meskipun terjadi kegagalan sistem, error, atau konkurensi akses.

\subsection{Atomicity (Atomisitas)}
\begin{enumerate}
    \item Transaksi harus dijalankan sepenuhnya atau dibatalkan sepenuhnya
    \item Tidak ada hasil parsial; jika satu operasi gagal, semua operasi dibatalkan
    \item Implementasi menggunakan mekanisme COMMIT dan ROLLBACK
    \item Contoh: Transfer uang antar rekening harus mengurangi saldo satu rekening dan menambah rekening lain, atau tidak melakukan perubahan sama sekali
\end{enumerate}

\subsection{Consistency (Konsistensi)}
\begin{enumerate}
    \item Transaksi harus membawa database dari satu state valid ke state valid lainnya
    \item Semua aturan integritas data (constraints, cascade, triggers) harus tetap terpenuhi
    \item Jumlah total saldo sebelum dan sesudah transaksi harus tetap sama
    \item Melindungi aplikasi dari data yang tidak konsisten
\end{enumerate}

\subsection{Isolation (Isolasi)}
\begin{enumerate}
    \item Transaksi harus berjalan seolah-olah tidak ada transaksi lain yang berjalan secara bersamaan
    \item Mencegah transaksi membaca data "kotor" dari transaksi lain yang belum commit
    \item PostgreSQL mengimplementasikan multi-version concurrency control (MVCC)
    \item Terdapat beberapa tingkat isolasi yang bisa dipilih sesuai kebutuhan
\end{enumerate}

\subsection{Durability (Durabilitas)}
\begin{enumerate}
    \item Perubahan yang sudah di-commit harus tetap tersimpan, bahkan jika terjadi kegagalan sistem
    \item PostgreSQL menggunakan Write-Ahead Logging (WAL) untuk menjamin durabilitas
    \item Data yang sudah di-commit harus bertahan dari crash, power failure, dll
    \item Memungkinkan pemulihan data setelah kegagalan sistem
\end{enumerate}

\section{Tingkat Isolasi Transaksi}

PostgreSQL mendukung beberapa tingkat isolasi sesuai dengan standar SQL. Setiap tingkat isolasi menawarkan keseimbangan antara konsistensi dan performa.

\subsection{READ UNCOMMITTED}
\begin{enumerate}
    \item PostgreSQL tidak benar-benar mengimplementasikan tingkat ini
    \item Dalam PostgreSQL, READ UNCOMMITTED sama dengan READ COMMITTED
    \item Tidak mungkin membaca perubahan yang belum di-commit (dirty read)
\end{enumerate}

\subsection{READ COMMITTED (Default)}
\begin{enumerate}
    \item Tingkat isolasi default di PostgreSQL
    \item Transaksi hanya dapat membaca data yang sudah di-commit
    \item Mencegah dirty read, tetapi memungkinkan non-repeatable read dan phantom read
    \item Membaca yang sama dalam satu transaksi dapat mengembalikan hasil berbeda jika transaksi lain melakukan commit
\end{enumerate}

\subsection{REPEATABLE READ}
\begin{enumerate}
    \item Transaksi hanya melihat data yang sudah di-commit sebelum transaksi dimulai
    \item Mencegah dirty read dan non-repeatable read
    \item Melindungi dari perubahan data yang dibaca, tetapi masih memungkinkan phantom read
    \item Semua query dalam satu transaksi melihat snapshot database yang sama
\end{enumerate}

\subsection{SERIALIZABLE}
\begin{enumerate}
    \item Tingkat isolasi tertinggi
    \item Mencegah semua anomali konkurensi (dirty read, non-repeatable read, phantom read)
    \item Transaksi berjalan seolah-olah dieksekusi secara serial (satu per satu)
    \item Dapat menyebabkan serialization error yang memerlukan retry
    \item Memberikan konsistensi tertinggi dengan trade-off performa
\end{enumerate}

\begin{figure}[h]
	\centering
	\begin{tabular}{|l|c|c|c|}
		\hline
		\textbf{Tingkat Isolasi} & \textbf{Dirty Read} & \textbf{Non-repeatable Read} & \textbf{Phantom Read} \\
		\hline
		READ UNCOMMITTED* & Tidak & Ya & Ya \\
		\hline
		READ COMMITTED & Tidak & Ya & Ya \\
		\hline
		REPEATABLE READ & Tidak & Tidak & Ya** \\
		\hline
		SERIALIZABLE & Tidak & Tidak & Tidak \\
		\hline
	\end{tabular}
	\caption{Tingkat Isolasi dan Anomali Konkurensi di PostgreSQL\\
	*Sama dengan READ COMMITTED di PostgreSQL\\
	**Di PostgreSQL, REPEATABLE READ juga mencegah phantom read, yang melebihi standar SQL}
\end{figure}

\subsection{Anomali Konkurensi}

Berikut adalah penjelasan lebih detail mengenai anomali yang dapat terjadi:

\begin{enumerate}
    \item \textbf{Dirty Read}: Membaca data yang belum di-commit oleh transaksi lain
    \item \textbf{Non-repeatable Read}: Transaksi membaca data dua kali, tetapi antara dua pembacaan tersebut, data diubah dan di-commit oleh transaksi lain
    \item \textbf{Phantom Read}: Transaksi mengeksekusi query yang mengembalikan himpunan baris, tetapi transaksi lain insert/delete baris yang memenuhi kondisi query
\end{enumerate}

\section{Strategi Penanganan Konkurensi}

Untuk menangani konkurensi dengan baik, PostgreSQL menyediakan beberapa strategi:

\subsection{Pessimistic Concurrency Control}
\begin{enumerate}
    \item Menggunakan locking untuk mencegah konflik
    \item Row-level locking dengan FOR UPDATE, FOR SHARE, dll
    \item Table-level locking dengan LOCK TABLE
    \item Mencegah konflik sebelum terjadi
    \item Dapat menyebabkan deadlock
\end{enumerate}

\subsection{Optimistic Concurrency Control}
\begin{enumerate}
    \item Mengasumsikan konflik jarang terjadi
    \item Tidak mengunci data, tetapi memeriksa apakah data berubah sebelum commit
    \item Biasanya menggunakan version number atau timestamp
    \item Cocok untuk lingkungan dengan read-heavy workload
    \item Memerlukan retry jika terjadi konflik
\end{enumerate}

\subsection{Multi-Version Concurrency Control (MVCC)}
\begin{enumerate}
    \item PostgreSQL menggunakan MVCC untuk implementasi isolasi
    \item Setiap transaksi melihat snapshot database pada titik waktu tertentu
    \item Memungkinkan reader tidak mengunci writer dan sebaliknya
    \item Meningkatkan konkurensi karena read tidak menghalangi write
    \item Trade-off adalah overhead penyimpanan untuk multiple versions
\end{enumerate}

\section{Tahapan Praktikum}

\subsection{Persiapan}
\begin{enumerate}
	\item Pastikan container Docker PostgreSQL sudah berjalan
	\item Gunakan pgAdmin atau PSQL untuk mengakses database
	\item Pastikan data sudah diimpor menggunakan script yang disediakan
\end{enumerate}

\subsection{Implementasi Transaksi Dasar}
Mari kita mulai dengan mempelajari cara mengimplementasikan transaksi dasar di PostgreSQL:

\begin{lstlisting}[language=SQL]
	-- Transaksi dasar
	BEGIN;
	UPDATE products SET stock_quantity = stock_quantity - 5 WHERE product_id = 1;
	INSERT INTO orders (customer_id, status) VALUES (1, 'pending');
	COMMIT;
	
	-- Transaksi dengan ROLLBACK
	BEGIN;
	UPDATE products SET stock_quantity = stock_quantity - 100 WHERE product_id = 1;
	-- Oops, stok tidak cukup!
	ROLLBACK;
\end{lstlisting}

\subsection{Demonstrasi Tingkat Isolasi Transaksi}
Selanjutnya, kita akan mendemonstrasikan perbedaan antara berbagai tingkat isolasi:

\subsubsection{READ COMMITTED (Default)}
Buka dua terminal PostgreSQL secara bersamaan dan jalankan:

Terminal 1:
\begin{lstlisting}[language=SQL]
	BEGIN;
	SET TRANSACTION ISOLATION LEVEL READ COMMITTED;
	UPDATE bank_accounts SET balance = balance - 1000000 WHERE account_id = 1;
	-- Tunggu beberapa saat sebelum commit
	SELECT pg_sleep(30);
	COMMIT;
\end{lstlisting}

Terminal 2:
\begin{lstlisting}[language=SQL]
	BEGIN;
	SET TRANSACTION ISOLATION LEVEL READ COMMITTED;
	-- Ini akan menampilkan nilai sebelum update di Terminal 1
	SELECT * FROM bank_accounts WHERE account_id = 1;
	-- Tunggu Terminal 1 commit
	SELECT pg_sleep(35);
	-- Ini akan menampilkan nilai setelah update di Terminal 1
	SELECT * FROM bank_accounts WHERE account_id = 1;
	COMMIT;
\end{lstlisting}

\subsubsection{REPEATABLE READ}
Buka dua terminal PostgreSQL dan jalankan:

Terminal 1:
\begin{lstlisting}[language=SQL]
	BEGIN;
	SET TRANSACTION ISOLATION LEVEL REPEATABLE READ;
	SELECT * FROM bank_accounts WHERE account_id = 1;
	-- Tunggu Terminal 2 melakukan update dan commit
	SELECT pg_sleep(30);
	-- Ini masih akan menampilkan nilai asli meskipun Terminal 2 sudah commit
	SELECT * FROM bank_accounts WHERE account_id = 1;
	COMMIT;
\end{lstlisting}

Terminal 2:
\begin{lstlisting}[language=SQL]
	BEGIN;
	UPDATE bank_accounts SET balance = balance - 1000000 WHERE account_id = 1;
	COMMIT;
\end{lstlisting}

\subsubsection{SERIALIZABLE}
Buka dua terminal PostgreSQL dan jalankan:

Terminal 1:
\begin{lstlisting}[language=SQL]
	BEGIN;
	SET TRANSACTION ISOLATION LEVEL SERIALIZABLE;
	SELECT SUM(balance) FROM bank_accounts WHERE customer_id = 1;
	-- Tunggu Terminal 2 menambahkan account baru
	SELECT pg_sleep(30);
	-- Ini tidak akan melihat account baru karena isolasi serializable
	SELECT SUM(balance) FROM bank_accounts WHERE customer_id = 1;
	COMMIT;
\end{lstlisting}

Terminal 2:
\begin{lstlisting}[language=SQL]
	BEGIN;
	INSERT INTO bank_accounts (customer_id, balance)
	VALUES (1, 5000000);
	COMMIT;
\end{lstlisting}

\subsection{Mengatasi Kondisi Konkurensi Tinggi}

\subsubsection{Menggunakan Lock (Pessimistic Concurrency Control)}
\begin{lstlisting}[language=SQL]
	-- Row-level lock
	BEGIN;
	SELECT * FROM products WHERE product_id = 1 FOR UPDATE;
	-- Sekarang row ini terkunci hingga transaksi selesai
	UPDATE products SET stock_quantity = stock_quantity - 5 WHERE product_id = 1;
	COMMIT;
	
	-- Lock pada tingkat tabel
	BEGIN;
	LOCK TABLE products IN EXCLUSIVE MODE;
	-- Operasi pada tabel products
	COMMIT;
\end{lstlisting}

\subsubsection{Optimistic Concurrency Control}
\begin{lstlisting}[language=SQL]
	-- Menggunakan last_updated sebagai version control
	BEGIN;
	SELECT product_id, stock_quantity, last_updated
	FROM products WHERE product_id = 1;
	
	-- Asumsikan kita mendapatkan last_updated = '2023-07-15 10:00:00'
	UPDATE products
	SET stock_quantity = stock_quantity - 5,
	    last_updated = CURRENT_TIMESTAMP
	WHERE product_id = 1 AND last_updated = '2023-07-15 10:00:00';
	
	-- Jika tidak ada baris yang terupdate, berarti data sudah berubah
	-- dan kita harus mengulang transaksi
	COMMIT;
\end{lstlisting}

\subsection{Implementasi Function Transfer Uang}
\begin{lstlisting}[language=SQL]
	-- Function sudah dibuat di 02-functions.sql
	-- Cobalah dari dua terminal berbeda
	BEGIN;
	SELECT * FROM bank_accounts WHERE account_id IN (1, 2);
	SELECT transfer_money(1, 2, 500000);
	SELECT * FROM bank_accounts WHERE account_id IN (1, 2);
	COMMIT;
\end{lstlisting}

\subsection{Pengujian Konkurensi Tinggi}
Untuk menguji bagaimana transaksi berperilaku pada konkurensi tinggi, gunakan script benchmark:

\begin{lstlisting}[language=bash]
	# Di terminal
	cd scripts
	bash benchmark.sh ../praktikum/praktikum2_transaction/transfer_money.sql 20 100 10
\end{lstlisting}

\subsection{Troubleshooting}
Jika mengalami error duplicate key, cek sequence id pada tabel tersebut kemudian reset ke max dari id tabel:

\begin{lstlisting}[language=SQL]
	-- tableName_columnName_seq
	SELECT currval('orders_order_id_seq');
	-- untuk reset
	SELECT setval('orders_order_id_seq', (SELECT MAX(order_id) FROM orders));
\end{lstlisting}

\section{Tugas Praktikum}

\begin{enumerate}
    \item Implementasikan sistem pemesanan produk dengan transaksi yang:
    \begin{itemize}
        \item Mengurangi stok produk
        \item Membuat pesanan baru
        \item Menambahkan item pesanan
        \item Memastikan stok mencukupi
        \item Harus ACID-compliant
    \end{itemize}
    
    \item Implementasikan dua strategi konkurensi berbeda dan bandingkan performanya:
    \begin{itemize}
        \item Row-level locking (pessimistic)
        \item Optimistic concurrency control
    \end{itemize}
    
    \item Lakukan pengujian konkurensi tinggi (minimal 20 koneksi simultan) dan analisis:
    \begin{itemize}
        \item Throughput (transactions per second)
        \item Deadlock yang terjadi (jika ada)
        \item Kesalahan konkurensi lainnya
    \end{itemize}
    
    \item Buat laporan yang menjelaskan bagaimana implementasi transaksi mempengaruhi konkurensi tinggi
\end{enumerate}

\section{Pertanyaan Diskusi}

\begin{enumerate}
    \item Apa perbedaan antara READ COMMITTED dan REPEATABLE READ? Kapan sebaiknya menggunakan masing-masing?
    \item Bagaimana cara mendeteksi dan mengatasi deadlock pada PostgreSQL?
    \item Apa trade-off antara optimistic dan pessimistic concurrency control?
    \item Bagaimana dampak tingkat isolasi transaksi terhadap performa sistem dengan konkurensi tinggi?
    \item Bagaimana strategi yang tepat untuk menangani sistem dengan write-heavy workload?
\end{enumerate}

\section{Referensi}

\begin{itemize}
    \item \href{https://www.postgresql.org/docs/current/transaction-iso.html}{PostgreSQL Documentation: Transactions}
    \item \href{https://www.postgresql.org/docs/current/mvcc.html}{PostgreSQL: Concurrency Control}
    \item \href{https://www.postgresql.org/docs/current/explicit-locking.html}{PostgreSQL: Explicit Locking}
    \item \href{https://www.postgresql.org/docs/current/functions-admin.html}{PostgreSQL: Administrative Functions}
    \item \href{https://www.postgresql.org/docs/current/monitoring-stats.html}{PostgreSQL: Monitoring Database Activity}
\end{itemize}