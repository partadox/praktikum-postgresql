
% Gunakan chapter* agar tidak bernomor
\chapter*{Persiapan Praktikum}
\addcontentsline{toc}{chapter}{Persiapan Praktikum}

\renewcommand{\thesection}{\arabic{section}}

\section{Deskripsi Praktikum}

Praktikum ini bertujuan untuk memahami konsep ACID (Atomicity, Consistency, Isolation, Durability) dan High Concurrency pada database PostgreSQL melalui empat skenario praktis.

Praktikum ini terdiri dari empat modul yang saling berkaitan:

\begin{enumerate}
    \item \textbf{Indexing}: Mempelajari cara mengoptimalkan performa query dengan index dan mengukur dampaknya pada konkurensi tinggi.

    \item \textbf{Transaction}: Mempelajari konsep ACID, tingkat isolasi transaksi, dan menangani konkurensi tinggi dengan benar.

    \item \textbf{Partitioning}: Mempelajari cara membagi tabel besar menjadi bagian yang lebih kecil dan dampaknya terhadap performa.

    \item \textbf{Backup dan Recovery}: Mempelajari strategi backup dan recovery untuk menjamin ketersediaan dan integritas data.
\end{enumerate}

%\section{Struktur Repositori}

\section{Persiapan Environment}

\subsection{Menggunakan Docker}

\begin{enumerate}
    \item Install Docker dan Docker Compose di komputer Anda
    \item Unduh file praktikum yang sudah disediakan oleh asisten praktikum
    \item Jalankan docker-compose file untuk membuat container dengan perintah:
\end{enumerate}

\begin{lstlisting}[language=bash]
docker-compose up -d
\end{lstlisting}

\begin{enumerate}
    \setcounter{enumi}{3}
    \item Generate data dummy:
\end{enumerate}

\begin{lstlisting}[language=bash]
cd scripts
pip install faker
python generate_data.py
\end{lstlisting}

\begin{enumerate}
    \setcounter{enumi}{4}
    \item Import data ke PostgreSQL:
\end{enumerate}

\begin{lstlisting}[language=bash]
cd scripts
bash import_data.sh
\end{lstlisting}

\begin{enumerate}
    \setcounter{enumi}{5}
    \item Akses pgAdmin di browser:
    \begin{itemize}
        \item URL: \url{http://localhost:5050}
        \item Email: admin@example.com
        \item Password: p4ssw0rd
    \end{itemize}

    \item Tambahkan server di pgAdmin:
    \begin{itemize}
        \item Name: PostgreSQL Praktikum
        \item Host: postgres
        \item Port: 5432
        \item Username: praktikan
        \item Password: p4ssw0rd
    \end{itemize}
\end{enumerate}

\section{Petunjuk Penggunaan Modul}

Setiap modul praktikum memiliki:
\begin{itemize}
    \item PDF Modul
    \item File README.md dengan penjelasan konsep dan langkah-langkah
    \item File SQL dengan query untuk latihan
    \item Tugas 
\end{itemize}

\section{Panduan Pengerjaan}

\begin{enumerate}
    \item Praktikum dikerjakan dalam kelompok 
    \item Setiap kelompok mengerjakan semua modul praktikum
    \item Pelaksanaan praktikum dilakukan oleh asisten praktikum dengan pengawasan koordinator dosen praktikum
    \item Setiap kelompok membuat laporan akhir berisi:
    \begin{itemize}
        \item Jawaban tugas dari setiap modul
        \item Analisis dan kesimpulan
    \end{itemize}
\end{enumerate}

\section{Struktur Folder Praktikum}

Folder praktikum memiliki struktur sebagai berikut:

\begin{lstlisting}[language=bash,basicstyle=\ttfamily\small]
	postgres-praktikum/
	|-- docker-compose.yml           
	|-- init/                         
	|   |-- 01-schema.sql            
	|   `-- 02-functions.sql        
	|-- data/                       
	|   |-- products.csv             
	|   |-- customers.csv           
	|   |-- orders.csv            
	|   `-- order_items.csv           
	|-- scripts/                    
	|   |-- generate_data.py          
	|   |-- import_data.sh           
	|   |-- benchmark.sh             
	|   `-- README.md               
	`-- praktikum/                   
	|		|-- praktikum1_indexing/      
	|		|-- praktikum2_transaction/ 
	|		|-- praktikum3_partitioning/ 
	`--	|-- praktikum4_backup/    
\end{lstlisting}

\section{Dataset Praktikum}

Dataset yang digunakan adalah data e-commerce dummy yang berisi:
\begin{itemize}
    \item 10.000 produk
    \item 5.000 pelanggan
    \item 20.000 pesanan
    \item 50.000 item pesanan
\end{itemize}

\section{Referensi}

\begin{itemize}
    \item \href{https://www.postgresql.org/docs/}{PostgreSQL Documentation}
    \item \href{https://www.postgresql.org/docs/current/mvcc.html}{PostgreSQL Concurrency}
    \item \href{https://www.postgresql.org/docs/current/indexes.html}{PostgreSQL Indexing}
    \item \href{https://www.postgresql.org/docs/current/ddl-partitioning.html}{PostgreSQL Partitioning}
    \item \href{https://www.postgresql.org/docs/current/backup.html}{PostgreSQL Backup and Restore}
\end{itemize}