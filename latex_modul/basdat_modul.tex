\documentclass[a4paper,12pt]{report}

% Paket-paket yang diperlukan
\usepackage[bahasa]{babel}
\usepackage{graphicx}
\usepackage{hyperref}
\usepackage{enumitem}
\usepackage{listings}
\usepackage{xcolor}
\usepackage{tcolorbox}
\usepackage{courier}
\usepackage{fancyhdr}
\usepackage{indentfirst}


% Package TikZ untuk diagram
\usepackage{tikz}
\usepackage{pgf}
\usetikzlibrary{mindmap}
\usetikzlibrary{arrows.meta}
\usetikzlibrary{positioning}
\usetikzlibrary{shapes,arrows}
\usetikzlibrary{er}
\usetikzlibrary{calc}

% Library tambahan untuk diagram kompleks
\usetikzlibrary{trees}
\usetikzlibrary{shadows}
\usetikzlibrary{backgrounds}
\usetikzlibrary{decorations}

% Atur penomoran section agar tidak terpengaruh chapter
% Ini akan memungkinkan penomoran section dimulai dari 1 di setiap chapter
\counterwithout{section}{chapter}

% Konfigurasi hyperref
\hypersetup{
    colorlinks=true,
    linkcolor=black,
    filecolor=magenta,      
    urlcolor=cyan,
    pdftitle={Modul Praktikum Sistem Manajemen Basis Data},
    pdfpagemode=FullScreen,
}

% Konfigurasi untuk kode program
\definecolor{codegreen}{rgb}{0,0.6,0}
\definecolor{codegray}{rgb}{0.5,0.5,0.5}
\definecolor{codepurple}{rgb}{0.58,0,0.82}
\definecolor{backcolour}{rgb}{0.95,0.95,0.92}

\lstdefinestyle{mystyle}{
    backgroundcolor=\color{backcolour},
    commentstyle=\color{codegreen},
    keywordstyle=\color{magenta},
    numberstyle=\tiny\color{codegray},
    stringstyle=\color{codepurple},
    basicstyle=\ttfamily\small,
    breakatwhitespace=false,
    breaklines=true,
    captionpos=b,
    keepspaces=true,
    numbers=left,
    numbersep=5pt,
    showspaces=false,
    showstringspaces=false,
    showtabs=false,
    tabsize=2
}

\lstset{style=mystyle}

% Konfigurasi header dan footer
\pagestyle{fancy}
\fancyhf{}
\fancyhead[L]{}
\fancyhead[R]{Departemen Teknik Komputer ITS}
\fancyfoot[C]{\thepage}
\renewcommand{\headrulewidth}{0.4pt}
\renewcommand{\footrulewidth}{0.4pt}

\begin{document}

% Modifikasi penomoran ToC untuk tidak menampilkan 0 di section
\renewcommand{\thesection}{\arabic{section}}

% Input Cover
% Informasi dokumen untuk cover
\title{\Huge \textbf{Modul Praktikum\\Basis Data: ACID dan High Concurrency\\dengan PostgreSQL}}
\author{Departemen Teknik Komputer\\Institut Teknologi Sepuluh Nopember}
\date{2025}

% Halaman cover
\begin{titlepage}
    \centering
    \vspace*{1cm}
    {\includegraphics[width=0.3\textwidth]{logo-tekkom.png}\par}
    \vspace{1.5cm}
    {\huge\bfseries Modul Praktikum\\Sistem Manajemen Basis Data\par}
    \vspace{1cm}
    {\Large  ACID dan High Concurrency PostgreSQL\par}
    \vspace{2cm}
    {\Large\itshape Departemen Teknik Komputer\\Institut Teknologi Sepuluh Nopember\\Surabaya\par}
    \vfill
    {\large Penulis: Arta Kusuma Hernanda\par}
    \vspace{0.5cm}
    % Footer
    {\large 2025\par}
\end{titlepage}

% Input Kata Pengantar
% Kata Pengantar
\chapter*{Kata Pengantar}
\addcontentsline{toc}{chapter}{Kata Pengantar}

Puji syukur kami panjatkan kepada Tuhan Yang Maha Esa atas segala rahmat dan karunia-Nya sehingga Modul Praktikum Basis Data: ACID dan High Concurrency dengan PostgreSQL ini dapat tersusun hingga selesai. Modul ini disusun untuk memenuhi kebutuhan mahasiswa dalam pembelajaran mata kuliah Basis Data di Departemen Teknik Komputer, Institut Teknologi Sepuluh Nopember Surabaya.

Modul praktikum ini bertujuan untuk memberikan pemahaman mendalam tentang konsep ACID (Atomicity, Consistency, Isolation, Durability) dan High Concurrency pada database PostgreSQL melalui empat skenario praktis. Melalui modul ini, mahasiswa diharapkan dapat memahami dan mempraktikkan berbagai teknik optimasi database seperti indexing, transactions, partitioning, serta strategi backup dan recovery.

Kami menyadari bahwa modul praktikum ini masih jauh dari kesempurnaan. Oleh karena itu, kritik dan saran yang membangun sangat kami harapkan demi perbaikan modul ini di masa yang akan datang.

Akhir kata, kami berharap modul praktikum ini dapat bermanfaat bagi mahasiswa dalam mengembangkan kemampuan dalam pengelolaan basis data dan dapat menjadi bekal yang berguna di dunia kerja nantinya.

\vspace{1cm}

\begin{flushright}
Surabaya, April 2025\\

Tim Penyusun
\end{flushright}

% Daftar Isi
\tableofcontents
\newpage

% Input Persiapan Praktikum (konten README.md)

% Gunakan chapter* agar tidak bernomor
\chapter*{Persiapan Praktikum}
\addcontentsline{toc}{chapter}{Persiapan Praktikum}

\renewcommand{\thesection}{\arabic{section}}

\section{Deskripsi Praktikum}

Praktikum ini bertujuan untuk memahami konsep ACID (Atomicity, Consistency, Isolation, Durability) dan High Concurrency pada database PostgreSQL melalui empat skenario praktis.

Praktikum ini terdiri dari empat modul yang saling berkaitan:

\begin{enumerate}
    \item \textbf{Indexing}: Mempelajari cara mengoptimalkan performa query dengan index dan mengukur dampaknya pada konkurensi tinggi.

    \item \textbf{Transaction}: Mempelajari konsep ACID, tingkat isolasi transaksi, dan menangani konkurensi tinggi dengan benar.

    \item \textbf{Partitioning}: Mempelajari cara membagi tabel besar menjadi bagian yang lebih kecil dan dampaknya terhadap performa.

    \item \textbf{Backup dan Recovery}: Mempelajari strategi backup dan recovery untuk menjamin ketersediaan dan integritas data.
\end{enumerate}

%\section{Struktur Repositori}

\section{Persiapan Environment}

\subsection{Menggunakan Docker}

\begin{enumerate}
    \item Install Docker dan Docker Compose di komputer Anda
    \item Unduh file praktikum yang sudah disediakan oleh asisten praktikum
    \item Jalankan docker-compose file untuk membuat container dengan perintah:
\end{enumerate}

\begin{lstlisting}[language=bash]
docker-compose up -d
\end{lstlisting}

\begin{enumerate}
    \setcounter{enumi}{3}
    \item Generate data dummy:
\end{enumerate}

\begin{lstlisting}[language=bash]
cd scripts
pip install faker
python generate_data.py
\end{lstlisting}

\begin{enumerate}
    \setcounter{enumi}{4}
    \item Import data ke PostgreSQL:
\end{enumerate}

\begin{lstlisting}[language=bash]
cd scripts
bash import_data.sh
\end{lstlisting}

\begin{enumerate}
    \setcounter{enumi}{5}
    \item Akses pgAdmin di browser:
    \begin{itemize}
        \item URL: \url{http://localhost:5050}
        \item Email: admin@example.com
        \item Password: p4ssw0rd
    \end{itemize}

    \item Tambahkan server di pgAdmin:
    \begin{itemize}
        \item Name: PostgreSQL Praktikum
        \item Host: postgres
        \item Port: 5432
        \item Username: praktikan
        \item Password: p4ssw0rd
    \end{itemize}
\end{enumerate}

\section{Petunjuk Penggunaan Modul}

Setiap modul praktikum memiliki:
\begin{itemize}
    \item PDF Modul
    \item File README.md dengan penjelasan konsep dan langkah-langkah
    \item File SQL dengan query untuk latihan
    \item Tugas 
\end{itemize}

\section{Panduan Pengerjaan}

\begin{enumerate}
    \item Praktikum dikerjakan dalam kelompok 
    \item Setiap kelompok mengerjakan semua modul praktikum
    \item Pelaksanaan praktikum dilakukan oleh asisten praktikum dengan pengawasan koordinator dosen praktikum
    \item Setiap kelompok membuat laporan akhir berisi:
    \begin{itemize}
        \item Jawaban tugas dari setiap modul
        \item Analisis dan kesimpulan
    \end{itemize}
\end{enumerate}

\section{Struktur Folder Praktikum}

Folder praktikum memiliki struktur sebagai berikut:

\begin{lstlisting}[language=bash,basicstyle=\ttfamily\small]
	postgres-praktikum/
	|-- docker-compose.yml           
	|-- init/                         
	|   |-- 01-schema.sql            
	|   `-- 02-functions.sql        
	|-- data/                       
	|   |-- products.csv             
	|   |-- customers.csv           
	|   |-- orders.csv            
	|   `-- order_items.csv           
	|-- scripts/                    
	|   |-- generate_data.py          
	|   |-- import_data.sh           
	|   |-- benchmark.sh             
	|   `-- README.md               
	`-- praktikum/                   
	|		|-- praktikum1_indexing/      
	|		|-- praktikum2_transaction/ 
	|		|-- praktikum3_partitioning/ 
	`--	|-- praktikum4_backup/    
\end{lstlisting}

\section{Dataset Praktikum}

Dataset yang digunakan adalah data e-commerce dummy yang berisi:
\begin{itemize}
    \item 10.000 produk
    \item 5.000 pelanggan
    \item 20.000 pesanan
    \item 50.000 item pesanan
\end{itemize}

\section{Referensi}

\begin{itemize}
    \item \href{https://www.postgresql.org/docs/}{PostgreSQL Documentation}
    \item \href{https://www.postgresql.org/docs/current/mvcc.html}{PostgreSQL Concurrency}
    \item \href{https://www.postgresql.org/docs/current/indexes.html}{PostgreSQL Indexing}
    \item \href{https://www.postgresql.org/docs/current/ddl-partitioning.html}{PostgreSQL Partitioning}
    \item \href{https://www.postgresql.org/docs/current/backup.html}{PostgreSQL Backup and Restore}
\end{itemize}

% Modul P1
\setcounter{section}{0}
\chapter{Implementasi Indexing pada PostgreSQL}
\setcounter{section}{0}
\section{Pendahuluan}
Index pada database adalah struktur data khusus yang mempercepat operasi pencarian dan pengambilan data. Analoginya seperti indeks pada buku yang membantu kita menemukan konten tertentu dengan cepat tanpa perlu membaca seluruh buku. Dalam PostgreSQL, index disimpan dalam struktur khusus yang mengoptimalkan pencarian berdasarkan kolom tertentu.

Implementasi indexing yang tepat merupakan salah satu strategi utama untuk meningkatkan performa database, terutama pada sistem dengan data besar dan tingkat konkurensi tinggi.

\subsection{Tujuan Praktikum}
Dilaksanakannya praktikum indexing, praktikan diharapkan mampu:
\begin{enumerate}
    \item Memahami konsep dasar indexing pada database PostgreSQL
    \item Mempelajari tipe-tipe index di PostgreSQL
    \item Mengimplementasikan index yang tepat untuk sebuah skenario
    \item Menganalisis dampak index terhadap performa query konkuren
\end{enumerate}

\section{Konsep Dasar Indexing}

\subsection{Apa itu Index?}
Index pada database bekerja mirip dengan indeks buku: membantu sistem menemukan data dengan cepat tanpa memeriksa seluruh tabel (table scan). Index menyimpan pointer ke baris data dalam tabel berdasarkan nilai kolom yang diindeks.

\subsection{Bagaimana Index Bekerja}

Ketika query dijalankan, PostgreSQL menganalisis apakah menggunakan index akan lebih efisien daripada melakukan table scan. Jika menggunakan index lebih efisien, PostgreSQL akan:
\begin{enumerate}
    \item Mencari data di index berdasarkan kondisi WHERE
    \item Menemukan referensi (pointer) ke baris data yang relevan
    \item Mengambil data dari tabel menggunakan pointer tersebut
\end{enumerate}

\subsection{Kapan Menggunakan Index}
Index sangat berguna pada:
\begin{enumerate}
    \item Kolom yang sering digunakan dalam klausa WHERE
    \item Kolom yang sering digunakan untuk JOIN antar tabel
    \item Kolom yang sering digunakan dalam klausa ORDER BY atau GROUP BY
    \item Tabel berukuran besar dengan query yang hanya mengambil sebagian kecil data
\end{enumerate}

\subsection{Kapan Tidak Menggunakan Index}
Index tidak selalu bermanfaat pada:
\begin{enumerate}
    \item Tabel kecil yang lebih efisien dilakukan sequential scan
    \item Kolom dengan kardinalitas rendah (nilai yang berbeda sedikit)
    \item Kolom yang jarang digunakan dalam query
    \item Tabel yang sering diupdate/dihapus (overhead pemeliharaan index)
\end{enumerate}

\subsection{Dampak Index pada Performa}
Index memberikan dampak pada:
\begin{enumerate}
    \item \textbf{SELECT}: Mempercepat query dengan mengurangi jumlah baris yang perlu diperiksa
    \item \textbf{INSERT}: Memerlukan overhead tambahan untuk memperbarui index
    \item \textbf{UPDATE}: Memerlukan overhead untuk memperbarui index jika kolom yang diindex berubah
    \item \textbf{DELETE}: Memerlukan overhead untuk memperbarui index
\end{enumerate}

\section{Jenis-jenis Index di PostgreSQL}

PostgreSQL menyediakan beberapa jenis index yang dapat dipilih sesuai dengan kebutuhan:

\subsection{B-tree Index}
\begin{enumerate}
    \item Index default di PostgreSQL
    \item Efisien untuk perbandingan dengan operator equality (=) dan range $(<, >, <=, >=, BETWEEN)$
    \item Mendukung urutan untuk ORDER BY
    \item Cocok untuk kebanyakan kasus penggunaan
\end{enumerate}

\subsection{Hash Index}
\begin{enumerate}
    \item Optimal hanya untuk operator equality (=)
    \item Lebih cepat dari B-tree untuk operasi equality sederhana
    \item Tidak mendukung range queries atau sorting
    \item Dirancang untuk tabel hash di memori
\end{enumerate}

\subsection{GiST Index (Generalized Search Tree)}
\begin{enumerate}
    \item Index untuk data geometri, text-search, dan data kompleks lainnya
    \item Fleksibel, dapat digunakan untuk data custom
    \item Mendukung nearest-neighbor searches
    \item Digunakan untuk full-text search, data geografis, dll
\end{enumerate}

\subsection{GIN Index (Generalized Inverted Index)}
\begin{enumerate}
    \item Dirancang untuk nilai yang memiliki multiple components (arrays, jsonb, text-search)
    \item Efisien untuk pencarian yang membutuhkan matching multiple values
    \item Cocok untuk kolom yang menyimpan data semi-structured
    \item Lebih lambat untuk operasi insert dibanding GiST
\end{enumerate}

\subsection{BRIN Index (Block Range INdex)}
\begin{enumerate}
    \item Dirancang untuk tabel sangat besar dengan data yang terurut secara natural
    \item Sangat kecil dan efisien untuk kolom seperti timestamp, ID sequential, dll
    \item Kinerja query lebih rendah dari B-tree tapi overhead penyimpanan jauh lebih kecil
\end{enumerate}

\subsection{SP-GiST Index (Space-Partitioned GiST)}
\begin{enumerate}
    \item Untuk data yang bisa dipartisi secara non-overlapping
    \item Baik untuk data hierarchical seperti rentang IP, geo-data
    \item Mendukung nearest-neighbor searches
\end{enumerate}

\begin{figure}[h]
	\centering
	\begin{tabular}{|l|p{10cm}|}
		\hline
		\textbf{Jenis Index} & \textbf{Karakteristik dan Penggunaan} \\
		\hline
		B-tree & Index default PostgreSQL. Efisien untuk operasi perbandingan (=, <, >, BETWEEN) dan mendukung pengurutan (ORDER BY). \\
		\hline
		Hash & Khusus untuk operasi equality (=). Lebih cepat dari B-tree untuk lookup sederhana, tidak mendukung range. \\
		\hline
		GiST/GIN & Untuk data kompleks seperti full-text search, data spasial, array, dan JSON. GIN lebih lambat untuk insert tapi lebih cepat untuk search. \\
		\hline
		BRIN & Untuk tabel besar dengan data terurut secara natural (seperti timestamp). Ukuran kecil, overhead rendah. \\
		\hline
	\end{tabular}
	\caption{Jenis-jenis Index di PostgreSQL dan Penggunaannya}
\end{figure}

\subsection{Benchmark pada Konkurensi Tinggi}

Pada situasi konkurensi tinggi, performa index dapat dipengaruhi oleh beberapa faktor:

\begin{itemize}
    \item \textbf{Contention}: Kompetisi antar koneksi untuk mengakses data yang sama
    \item \textbf{Lock contention}: Waktu tunggu karena lock pada baris atau tabel
    \item \textbf{Index bloat}: Overhead karena index yang jarang di-maintenance
\end{itemize}

\section{Best Practices Implementasi Indexing}

\begin{enumerate}
    \item \textbf{Index kolom yang sering digunakan dalam WHERE, JOIN, ORDER BY}
    \item \textbf{Hindari over-indexing}: Terlalu banyak index berdampak negatif pada performa INSERT/UPDATE/DELETE
    \item \textbf{Gunakan composite index} untuk query dengan multiple conditions
    \item \textbf{Perhatikan urutan kolom} pada composite index (most selective first)
    \item \textbf{Pertimbangkan partial index} untuk subset data yang sering diakses
    \item \textbf{Gunakan EXPLAIN ANALYZE} untuk validasi penggunaan index
    \item \textbf{Jalankan VACUUM ANALYZE} secara berkala untuk memperbarui statistik
    \item \textbf{Monitor ukuran dan penggunaan index} menggunakan pg\_stat\_*
    \item \textbf{Lakukan REINDEX} pada index yang terfragmentasi
    \item \textbf{Evaluasi trade-off} antara kecepatan query vs overhead pemeliharaan
\end{enumerate}

\section{Tahapan Praktikum}

\subsection{Persiapan}
\begin{enumerate}
	\item Pastikan container Docker PostgreSQL sudah berjalan
	\item Gunakan pgAdmin atau PSQL untuk mengakses database
	\item Pastikan data sudah diimpor menggunakan script yang disediakan
\end{enumerate}

\subsection{Mengeksplorasi Execution Plan}
Pertama, Anda akan mempelajari cara melihat execution plan query menggunakan EXPLAIN dan EXPLAIN ANALYZE:

\begin{lstlisting}[language=SQL]
	-- Menampilkan execution plan
	EXPLAIN SELECT * FROM products WHERE category = 'Elektronik';
	
	-- Menampilkan execution plan beserta runtime
	EXPLAIN ANALYZE SELECT * FROM products WHERE category = 'Elektronik';
\end{lstlisting}

\subsection{Mengukur Performa Query Tanpa Index}
\begin{enumerate}
	\item Jalankan query berikut tanpa index dan catat waktu eksekusinya:
	
	\begin{lstlisting}[language=SQL]
		-- Query 1: Filter berdasarkan kategori
		SELECT COUNT(*) FROM products WHERE category = 'Elektronik';
		
		-- Query 2: Filter berdasarkan range harga
		SELECT * FROM products WHERE price BETWEEN 1000000 AND 5000000;
		
		-- Query 3: Filter berdasarkan tanggal pesanan
		SELECT o.order_id, o.customer_id, o.order_date 
		FROM orders o 
		WHERE o.order_date BETWEEN '2023-06-01' AND '2023-06-30';
		
		-- Query 4: Join tanpa index
		SELECT c.first_name, c.last_name, o.order_id, o.order_date, o.total_amount
		FROM customers c 
		JOIN orders o ON c.customer_id = o.customer_id 
		WHERE c.city = 'Jakarta';
		
		-- Query 5: Aggregate yang berat
		SELECT p.category, COUNT(*) as total_products, AVG(p.price) as avg_price
		FROM products p
		GROUP BY p.category
		ORDER BY total_products DESC;
	\end{lstlisting}
	
	\item Lakukan benchmark konkuren tanpa index:
	
	\begin{lstlisting}[language=bash]
		# Di terminal
		bash query_before.sql 20 100 10
	\end{lstlisting}
\end{enumerate}

\subsection{Implementasi Index yang Tepat}
Sekarang, tambahkan index yang sesuai untuk setiap query:

\begin{lstlisting}[language=SQL]
	-- Hapus index yang sudah ada (untuk tujuan praktikum)
	DROP INDEX IF EXISTS idx_products_category;
	DROP INDEX IF EXISTS idx_products_price;
	DROP INDEX IF EXISTS idx_orders_date;
	DROP INDEX IF EXISTS idx_customers_city;
	
	-- Index 1: B-Tree index untuk kategori produk
	CREATE INDEX idx_products_category ON products(category);
	
	-- Index 2: B-Tree index untuk range harga
	CREATE INDEX idx_products_price ON products(price);
	
	-- Index 3: B-Tree index untuk tanggal pesanan
	CREATE INDEX idx_orders_date ON orders(order_date);
	
	-- Index 4: Index untuk city pada tabel customers
	CREATE INDEX idx_customers_city ON customers(city);
	
	-- Index 5: Composite index untuk JOIN operation
	CREATE INDEX idx_orders_customer_id ON orders(customer_id);
\end{lstlisting}

\subsection{Mengukur Performa Setelah Indexing}
\begin{enumerate}
	\item Jalankan kembali query yang sama dan bandingkan waktu eksekusinya:
	
	\begin{lstlisting} [language=SQL]
		-- Jalankan semua query yang sama seperti sebelumnya dan bandingkan execution plan
		EXPLAIN ANALYZE SELECT COUNT(*) FROM products WHERE category = 'Elektronik';
		-- dan seterusnya untuk query lainnya
	\end{lstlisting}
	
	\item Lakukan benchmark konkuren dengan index:
	
	\begin{lstlisting} [language=bash]
		# Di terminal
		cd scripts
		bash benchmark.sh ../praktikum/praktikum1_indexing/query_after.sql 20 100 10
	\end{lstlisting}
\end{enumerate}

\subsection{Mempelajari Tipe Index Lain}
PostgreSQL mendukung beberapa tipe index. Cobalah implementasi berikut:

\begin{lstlisting} [language=SQL]
	-- Index Partial untuk produk dengan stok rendah
	CREATE INDEX idx_products_low_stock ON products(product_id, stock_quantity) 
	WHERE stock_quantity < 10;
	
	-- Index menggunakan ekspresi untuk pencarian case-insensitive
	CREATE INDEX idx_products_name_lower ON products(LOWER(name));
	
	-- BRIN Index untuk data berurutan (seperti timestamp)
	CREATE INDEX idx_orders_date_brin ON orders USING BRIN(order_date);
	
	-- GIN Index untuk pencarian full-text (jika menggunakan PostgreSQL >= 9.6)
	CREATE INDEX idx_products_description_gin ON products 
	USING GIN(to_tsvector('english', description));
\end{lstlisting}

\section{Referensi}

\begin{itemize}
    \item \href{https://www.postgresql.org/docs/current/indexes.html}{PostgreSQL Documentation: Indexes}
    \item \href{https://www.postgresql.org/docs/current/using-explain.html}{PostgreSQL Documentation: EXPLAIN}
    \item \href{https://www.postgresql.org/docs/current/performance-tips.html}{PostgreSQL Documentation: Performance Tips}
    \item \href{https://www.postgresql.org/docs/current/indexes-types.html}{PostgreSQL Documentation: Index Types}
    \item \href{https://www.postgresql.org/docs/current/pgbench.html}{PostgreSQL Documentation: pgbench}
\end{itemize}

\end{document}